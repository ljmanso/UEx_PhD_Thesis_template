\chapter{Otro capítulo}

\par
Este template tiene algunas cosas curiosas. Una de ellas es el comando \importante{importante}, que marca un término en negrita, lo incluye en el índice de términos y genera un símbolo $\surd$ en el margen. Cuando hacemos \textit{make} en el directorio del template, además de compilarse la tesis se nos genera una gráfica en el fichero \textit{thesisometer.png} con el número de palabras y de ``cosas por hacer'' que llevamos. Estas cosas por hacer se definen usando el comando \importante{PORHACER}, que hace que lo que se le ponga como parámetro sólo se muestre en modo \textit{draft} y que se registre en el fichero thesisometer\_tickets.txt para incluirse luego en la gráfica. 

\par
Lorem ipsum dolor sit amet, consectetur adipisicing elit, sed do eiusmod tempor incididunt ut labore et dolore magna aliqua. Ut enim ad minim veniam, quis nostrud exercitation ullamco laboris nisi ut aliquip ex ea commodo consequat. Duis aute irure dolor in reprehenderit in voluptate velit esse cillum dolore eu fugiat nulla pariatur. Excepteur sint occaecat cupidatat non proident, sunt in culpa qui officia deserunt mollit anim id est laborum.

\par
Lorem ipsum dolor sit amet, consectetur adipisicing elit, sed do eiusmod tempor incididunt ut labore et dolore magna aliqua. Ut enim ad minim veniam, quis nostrud exercitation ullamco laboris nisi ut aliquip ex ea commodo consequat. Duis aute irure dolor in reprehenderit in voluptate velit esse cillum dolore eu fugiat nulla pariatur. Excepteur sint occaecat cupidatat non proident, sunt in culpa qui officia deserunt mollit anim id est laborum.

